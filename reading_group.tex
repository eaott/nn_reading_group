\documentclass{article}
\usepackage[top=.5in, bottom=.5in, left=.9in, right=.9in]{geometry}
\usepackage[latin1]{inputenc}
\usepackage{enumerate}
\usepackage{hyperref}
\usepackage{graphics}
\usepackage{graphicx}
\usepackage{caption}
\usepackage{subcaption}
\usepackage{tabularx}
\usepackage{amsmath}
\usepackage{amssymb}
\usepackage{siunitx}
\usepackage{mathtools}
\usepackage{bbm}




% Use instead of natbib
\usepackage[backend=bibtex,citestyle=authoryear-comp,natbib=true,sorting=none,hyperref=true,maxnames=4,arxiv=pdf]{biblatex}
\renewbibmacro{in:}{}
\addbibresource{/Users/Evan/GitProjects/biomedical/references.bib}




\newcommand{\obar}[1]{\ensuremath{\overline{ #1 }}}
\newcommand{\iid}{\ensuremath{\stackrel{\textrm{iid}}{\sim}}}
\newcommand{\op}[2]{{\ensuremath{\underset{ #2 }{\operatorname{ #1 }}~}}}
\newcommand{\norm}[1]{{ \ensuremath{ \left\lVert  #1 \right\rVert  }  }}
\newcommand{\cov}{ \ensuremath{ \textrm{cov} } }
\newcommand{\var}{ \ensuremath{ \textrm{var} } }
\newcommand{\tr}{ \ensuremath{ \textrm{trace} } }
\newcommand{\df}{ \ensuremath{ \textrm{df} } }
\newcommand{\R}{ \ensuremath{ \mathbb{R} }}
\newcommand{\indicator}[1]{ \ensuremath{ \mathbbm{1}\left\{ #1 \right\} }   }

\usepackage{xcolor}
\definecolor{darkgreen}{rgb}{0,0.25,0}
\newcommand{\soln}{{\color{red}\textbf{Solution:~}\color{black}}}


\usepackage[formats]{listings}
\lstdefineformat{R}{~={\( \sim \)}}
\lstset{% general command to set parameter(s)
basicstyle=\small\ttfamily, % print whole listing small
keywordstyle=\bfseries\rmfamily,
keepspaces=true,
% underlined bold black keywords
commentstyle=\color{darkgreen}, % white comments
stringstyle=\ttfamily, % typewriter type for strings
showstringspaces=false,
numbers=left, numberstyle=\tiny, stepnumber=1, numbersep=5pt, %
frame=shadowbox,
rulesepcolor=\color{black},
,columns=fullflexible,format=R
} %
\renewcommand{\ttdefault}{cmtt}
% enumerate is numbered \begin{enumerate}[(I)] is cap roman in parens
% itemize is bulleted \begin{itemize}
% subfigures:
% \begin{subfigure}[b]{0.5\textwidth} \includegraphics{asdf.jpg} \caption{} \label{subfig:asdf} \end{subfigure}
\hypersetup{colorlinks=true, urlcolor=blue, linkcolor=blue, citecolor=blue}


\graphicspath{ {C:/Users/Evan/Desktop/} }
\title{\vspace{-6ex}NN/VI/Autoencoder Reading Group Notes\vspace{-2ex}} lorem
\author{Evan Ott \\ \href{mailto:evan.ott@utexas.edu}{evan.ott@utexas.edu}\vspace{-2ex}}
\date{Last updated: \today}
\setcounter{secnumdepth}{0}

\usepackage[parfill]{parskip}

\newcommand{\mycite}[1]{\item[\cite{#1}] ~\\ \fullcite{#1} \\ ~ \\}

\begin{document}
\maketitle

\section{Outline}
\begin{itemize}
  \item Variational inference (James)
  \item Stochastic variational inference (Michael)
  \item NN and backpropagation (Yuguang)
  \item Introduction to TensorFlow (Mo and Evan) -- actual hello world (like a single layer perceptron or a linear model) -- leave out complicated things like dropout, etc.
  \item Classical autoencoders (Mauricio)
  \item Variational autoencoders (Jennifer)
\end{itemize}

We'll have a GitHub page / Google Doc for papers, examples, talks, code, etc.

\section{2017-09-11: Variational Inference (James)}
\subsection{``Variational Bayes for Idiots''}
Today is a basic introduction to variational inference.

Observed data $X$ and hidden variables $Z$ (could be parameters for the whole dataset
-- like means and variances of mixture components of a GMM -- or the per-data-point hidden variable -- like an indicator
for each datapoint being in a cluster).

\begin{align*}
  p(x,z)&= p(x|z)\cdot p(z)
  \\
  \textrm{Bayes:~} p(z|x)&=\frac{p(x|z)p(z)}{p(x)}
  \\
  \textrm{\color{red}Problem:~\color{black}} p(x)=\int_Z p(x,z)dz
\end{align*}
$p(x)$ is challenging or impossible to compute.

Idea:
\begin{itemize}
  \item posit some family of approximations, $\mathcal{Q}$
  \item Find the member of $\mathcal{Q}$ that is ``closest'' to $p(z|x)$ in KL-divergence (or other measure)

  \hspace{5ex} Formally:
  \begin{equation}
    q^*(z)=\arg\min_{q(z)\in\mathcal{Q}}KL \left(q(z) \middle\| p(z|x)\right)
  \end{equation}
  where
  \begin{align*}
    \label{eq:KL}
    KL \left(q(z) \middle\| p(z|x)\right) &= \mathbb{E}_q(z)\left[\log\frac{q(z)}{p(z|x)}\right]
    \\
    &= \mathbb{E}_q\left[\log q(z)\right] - \mathbb{E}_q\left[\log p(z|x)\right]
  \end{align*}
\end{itemize}

This is a calculus of variations problem (like the brachistochrome problem), where this will eventually
lead to a vector optimization problem that's standing for a functional optimization problem.

This also will give us point estimates that are often better than running a Gibbs sampler, and be faster.
KL will favor putting probability mass where $p(z|x)$ is large, so that the variance of the approximated
distribution is often smaller than the variance of the true posterior. If we flipped the ratio in equation
\ref{eq:KL}, we would instead be looking at Expectation Propagation.

\subsubsection{Example: One-Hot Encoding}
\begin{align*}
  (x_i|\mu, c_i) &\sim N(c_i^\top \mu, 1)
  \\
  \textrm{e.g.,~}c_i &= (0, 0, 1, 0, 0) \in \{0,1\}^k
  \\
  \mu\in\mathbb{R}^k
  \\
  \mu_k&\sim N(0,\tau^2)
\end{align*}
Joint of data, parameters:
\begin{equation*}
  p(\mu,c,x)=p(\mu)\prod_{i=1}^N p(c_i)p(x_i|\mu,c_i)
\end{equation*}
Marginal or ``Evidence'': $p(x)=\int_{\mu,c}p(x,\mu,c)p(\mu,c)d\mu dc$

\subsubsection{ELBO}
ELBO: Evidence lower bound.

$\mathcal{Q}$: family of approximations.

Each $q(z)\in\mathcal{Q}$ is a candidate.

$\arg\min_{q(z)\in\mathcal{Q}}KL \left(q(z) \middle\| p(z|x)\right)$ is not computable:

\begin{align*}
  \mathbb{E}_q\left[\log q(z)\right] - \mathbb{E}_q\left[\log p(z|x)\right] &= \mathbb{E}_q\left[\log q(z)\right] - \mathbb{E}_q\left[\log p(x,z) - \log p(x)\right]
\end{align*}
But we can't compute $p(x)$, so we can't compute it, but the term is not dependent on the choice of $z$. So, we ignore that term,
and want to maximize its negative (the ELBO):
\begin{align*}
  ELBO(q)&=\mathbb{E}_q\left[\log p(x,z)\right]-\mathbb{E}_q\left[\log q(z)\right]
  \\
  &=-KL \left(q(z) \middle\| p(z|x)\right) + \log(p(x))
  \\
  &\leq \log(p(x))
\end{align*}
So we can re-write this as:
\begin{align*}
  ELBO(q)&=\mathbb{E}_q\left[\log p(x|z)\right]+\mathbb{E}_q\left[\log p(z)\right]-\mathbb{E}_q\left[\log q(z)\right]
  \\
  &=\mathbb{E}_q\left[\log p(x|z)\right]+KL\left(q(z)\middle\|p(z)\right)
\end{align*}
So we're making the likelihood as large as possible, and making the
approximate distribution that penalizes moving away from the prior. In other words,
be close to the data, and don't stray far from the prior.

\subsection{Note}
Carlos asked a question about not having to do the expectation of the log posterior with respect to $q$ to get
something more like a MAP estimate. But a reasonable compromise is doing a maximum marginal a posteriori estimate
where we marginalize over the local variables in the model (the cluster assignments in a GLM), but keep the global
parameters to find the MAP. That's a nice sort of Bayesian thing to do.

\subsection{Mean-field family}
This is a family of approximations where the correlation structure is completely independent.

\begin{equation*}
  \mathcal{Q}=\left\{q(z) : q(z)=\prod_{j=1}^M q_j(z_j)\right\}
\end{equation*}
This is not a model of the data (there's no $x$ in that). But we're going to connect it to the data
through the ELBO. Now, we have options for each $q_j$, where we might take a nice parametric form. We might have Gaussians for location parameters
or inverse gammas for scale parameters. But for some models where the complete conditionals are exponential, we can find optimal approximation to take for each $q_j$.

\subsection{Coordinate ascent}
Not totally different from Gibbs sampling, only we're doing optimization instead of probabilistic draws. We'll do
one latent variable at a time, such as all the components one after another, then each mean in a GMM. Then, start over.

Consider the $j$th latent variable $z_j$ with complete conditional $p(z_j|z_{-j}, x)$. With $z_{-j}$ fixed,
the optimal $q_j$ (that makes ELBO as large as possible) is of the form:

\begin{equation}
  q_j^*(z_j)\propto \exp\left\{ \mathbb{E}_{-j}\left[ \log p(z_j | z_{-j}, x)\right] \right\}
\end{equation}
Or, can write in terms of the joint (the constants will go away). The expectation is with
respect to the variational density over all the other parameters, i.e.,
\begin{equation*}
  z_{-j}\sim \prod_{l\neq j} q_l(z_l)
\end{equation*}

So, we set $q_j^*(z_j)$ for each $j$, cycle through, update, etc. until it converges.
% \begin{description}
% \mycite{}
% \end{description}



% If using biblatex
% \printbibliography

\end{document}
